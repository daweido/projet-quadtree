%pdflatex main.tex
\documentclass[12pt]{article}
\usepackage{natbib}
\usepackage[francais]{babel}
\usepackage{natbib}
\usepackage{url}
\usepackage[utf8x]{inputenc}
\usepackage{amsmath}
\usepackage{graphicx}
\usepackage{float}
\graphicspath{{images/}}
\usepackage{amsthm}
\usepackage{parskip}
\usepackage{fancyhdr}
\usepackage{xfrac}
\usepackage{esvect}
\usepackage{vmargin}
\usepackage{gensymb}
\setmarginsrb{3 cm}{2.5 cm}{3 cm}{2.5 cm}{1 cm}{1.5 cm}{1 cm}{1.5 cm}

\title{Projet Quadtree}								% Title
\author{Groupe 2}								% Author
\date{\today}											% Date

\makeatletter
\let\thetitle\@title
\let\theauthor\@author
\let\thedate\@date
\makeatother

\pagestyle{fancy}
\fancyhf{}
\rhead{\theauthor}
\lhead{\thetitle}
\cfoot{\thepage}
\newtheorem*{rappel}{Rappel}
\newtheorem*{nota}{N.B}
\newtheorem*{req}{Remarque}
\begin{document}

%%%%%%%%%%%%%%%%%%%%%%%%%%%%%%%%%%%%%%%%%%%%%%%%%%%%%%%%%%%%%%%%%%%%%%%%%%%%%%%%%%%%%%%%%

\begin{titlepage}
	\centering
    \vspace*{0.5 cm}
    \includegraphics[scale = 0.75 ]{logo1.png}\\[1.0 cm]	% University Logo
    \textsc{\LARGE EISTI}\\[2.0 cm]			% University Name
    \rule{\linewidth}{0.2 mm} \\[0.5 cm]
    { \huge \bfseries \thetitle}\\
    \rule{\linewidth}{0.2 mm} \\[1.5 cm]
	\textsc{\Large Algorithmique et programmation fonctionnelle}\\[0.5 cm]	% Course Code
	\textsc{\large CPI1 C2 - Groupe 2}\\[0.5 cm]		% Course Name
	
	\begin{minipage}{0.4\textwidth}
	\centering
		\begin{center} \large
		Amine TAGHROUT\\
		David RIGAUX \\
		Mehdi DALAA 
			\end{center}
			\end{minipage}~
			\begin{minipage}{0.4\textwidth}
	\end{minipage}\\[0.8 cm]
	{\large \thedate}\\[1 cm]
	\vfill
	
\end{titlepage}

%%%%%%%%%%%%%%%%%%%%%%%%%%%%%%%%%%%%%%%%%%%%%%%%%%%%%%%%%%%%%%%%%%%%%%%%%%%%%%%%%%%%%%%%%

\tableofcontents
\addtocontents{toc}{~\hfill\textbf{Page}\par}
\pagebreak

%%%%%%%%%%%%%%%%%%%%%%%%%%%%%%%%%%%%%%%%%%%%%%%%%%%%%%%%%%%%%%%%%%%%%%%%%%%%%%%%%%%%%%%%%

\section*{Introduction}
\addcontentsline{toc}{section}{\protect\numberline{}Introduction}
\section{Étude du projet}
Expliquer le but du projet comment on a procédé pour la résolution du projet
\section{LablGTK2}
Explication de la librairie graphique LablGTK2
\section{Arbre Quadtree}
Qu'est ce qu'un arbre Quadtree et comment on l'a créé
\section{Fonction I/O}
Explication qu'on doit charger et enregistrer des images
\subsection{Chargement}
Comment on a procédé pour le chargement
\subsection{Enregistrement}
Comment on a procédé pour l'enregistrement
\section{Fonctions de manipulation}
\subsection{Opérations Simples}
\subsubsection{Rotation de 90$^{\circ}$}
\subsubsection{Miroir}
\subsubsection{Inversion}
\subsection{Opérations Avancées}
\subsubsection{Compression}
\subsubsection{Segmentation}
\section{Graphiques}
Explication sur comment on a créé le GUI
\section*{Conclusion}
\addcontentsline{toc}{section}{\protect\numberline{}Conclusion}
\end{document}
